\documentclass[hidelinks,12pt]{article}

\usepackage{setspace}
\usepackage{lipsum}
\usepackage{geometry} % Optional: For better control of page margins

\usepackage{amssymb,amsmath,amsfonts,float,eurosym,geometry,ulem,graphicx,color,setspace,sectsty,comment,natbib,pdflscape,subfigure,array,booktabs}

\usepackage[absolute,overlay]{textpos} % Allows absolute positioning
\usepackage[bottom]{footmisc}
\usepackage{afterpage}

\usepackage{hyperref}
\hypersetup{
    colorlinks=true,
    linkcolor=blue,
    filecolor=blue,      
    urlcolor=blue,
    citecolor=blue,
    pdftitle={Overleaf Example},
    pdfpagemode=FullScreen,
    }
\urlstyle{same}

\usepackage{graphicx}
\usepackage{ragged2e}
\usepackage{adjustbox}
\usepackage{booktabs}
\usepackage{dcolumn}

\usepackage[style=apa, sorting=nyt]{biblatex}
\addbibresource{cite.bib}

\usepackage{booktabs}
\usepackage[flushleft]{threeparttable}

\usepackage[flushleft]{caption}
\newcommand\fnote[1]{\captionsetup{font=footnotesize}\caption*{#1}}

\usepackage[skip=10pt plus1pt, indent=0pt]{parskip}
\usepackage[capposition=top]{floatrow}

\newcommand{\tabnotes}[2]{\bottomrule \multicolumn{#1}{@{}p{0.70\linewidth}@{}}{\footnotesize #2 }\end{tabular}\end{table}}

\usepackage [english]{babel}
\usepackage [autostyle, english = american]{csquotes}
\MakeOuterQuote{"}

\normalem

\onehalfspacing
\newtheorem{theorem}{Theorem}
\newtheorem{corollary}[theorem]{Corollary}
\newtheorem{proposition}{Proposition}
\newenvironment{proof}[1][Proof]{\noindent\textbf{#1.} }{\ \rule{0.5em}{0.5em}}

\newtheorem{hyp}{Hypothesis}
\newtheorem{subhyp}{Hypothesis}[hyp]
\renewcommand{\thesubhyp}{\thehyp\alph{subhyp}}

\newcommand{\red}[1]{{\color{red} #1}}
\newcommand{\blue}[1]{{\color{blue} #1}}

\newcolumntype{L}[1]{>{\raggedright\let\newline\\arraybackslash\hspace{0pt}}m{#1}}
\newcolumntype{C}[1]{>{\centering\let\newline\\arraybackslash\hspace{0pt}}m{#1}}
\newcolumntype{R}[1]{>{\raggedleft\let\newline\\arraybackslash\hspace{0pt}}m{#1}}

\renewcommand{\arraystretch}{1.3}

\geometry{left=1.0in,right=1.0in,top=1.0in,bottom=1.0in}

%%% for flowcharts
\usepackage{tikz}
\usetikzlibrary{shapes.geometric, arrows}
\tikzstyle{startstop} = [rectangle, rounded corners, text width=8.5cm, minimum width=3cm, minimum height=1cm,text centered, draw=black, fill=white]
\tikzstyle{notes} = [rectangle, rounded corners, text width=4cm, minimum height=1cm, align=right, fill=white]
\tikzstyle{process} = [rectangle, rounded corners, text width=4cm,  minimum height=1cm, text centered, draw=black, fill=white]
\tikzstyle{decision} = [diamond, text width=2.5cm, text centered, draw=black, fill=white]
\tikzstyle{arrow} = [thick,->,>=stealth]

\begin{document}


\begin{singlespace}

\begin{titlepage}
\title{Do Bureaucrats Support Policies While Undermining Implementation? Evidence from Teachers’ Response to an Education Reform in Tanzania}
%\title{Do Policies Shift Bureaucrat Beliefs? Evidence from Education Reforms and a Teacher Experiment}

\author{Frank Odhiambo \thanks{Development Economics, University of G\"ottingen, Germany. E-mail: \href{mailto:frank.odhiambo@wiwi.uni-goettingen.de}{frank.odhiambo@wiwi.uni-goettingen.de}.}}

\date{\today}
\maketitle
\begin{abstract}
\begin{singlespace}
This paper investigates bureaucrats’ responses to top-down policy reforms, drawing on evidence from Tanzania’s nationwide free primary education (FPE) reform. Using a difference-in-differences design, I find that the reform increases teacher support for FPE by 18 percentage points. Yet this support coincides with an adverse behavioral response: rent-seeking. Over half of eligible parents report paying a bribe to secure school placement for their child, and public perceptions of teacher corruption rise by 32 percentage points. These results suggest that bureaucrats can simultaneously endorse a policy in principle while undermining its implementation in practice, challenging the conventional assumption that stated support guarantees fidelity. The findings highlight the need to consider bureaucratic incentives beyond attitudinal alignment when designing large-scale reforms. More broadly, the study contributes to understanding the interplay between bureaucrat attitudes and their behavioral responses to policy reforms, particularly in developing-country contexts.\\
\end{singlespace}

\textbf{Keywords:} Policy implementation, bureaucratic behavior, education reform, corruption, Tanzania \\

JEL Classification: H11, I21, D73, O15

\bigskip
\end{abstract}
\setcounter{page}{0}
\thispagestyle{empty}
\end{titlepage}
\pagebreak \newpage

\onehalfspacing

\section{Introduction}
\hspace{1em} The implementation success of many reforms often depends on the actions of frontline bureaucrats who translate policy directives into practice. While policymakers design reforms with specific objectives, the ultimate success of these interventions depend on whether implementing agents adopt behaviors consistent with policy goals. This paper examines this implementation challenge in the context of a nationwide reform that eliminated primary school fees in Tanzania. I examine how teachers as the key bureaucratic actors in schools responded to the reform, both in stated support and behaviorally.

I build on standard principal-agent models where policymakers (principals) design reforms that must be implemented by bureaucrats (agents) operating under different constraints and incentives. Policymakers, such as legislators and senior government officials, prioritize broad policy objectives, whereas bureaucrats operate under local constraints and hold discretion over day-to-day implementation. In education, teachers and school administrators embody the agent role, exercising influence that can either advance or undermine policy goals depending on how their beliefs and incentives align with those of the principal.

Existing research establishes that bureaucrat beliefs significantly influence implementation outcomes. For example, in educational settings, teachers' attitudes and expectations affect student achievement and educational choices (\cite{sabarwal_teacher_2022,fives_teachers_2016,liou_climate_2021}) with particularly strong evidence on how gender stereotypes shape course selection and long-term human capital formation (\cite{lavy_gender_2008,lavy_origins_2018}). More broadly, studies from various institutional contexts demonstrate that pre-existing norms and enforcement capacity determine compliance behavior. For instance, \textcite{fisman_corruption_2007} show how cultural norms affected diplomatic parking violations and \textcite{olken_monitoring_2007} documents how monitoring intensity reduces corruption in Indonesian infrastructure projects. These studies suggest that attitudes are important for implementation fidelity, but that incentives and constraints may also determine whether attitudes translate into compliant behavior.

More recently, economists have examined how policy actors update beliefs in response to new information. \textcite{vivalt_how_2023} show that policymakers and bureaucrats hold systematically different priors and that new evidence can shift those priors, with heterogeneous responsiveness. In related work from Brazil, \textcite{hjort_how_2021} demonstrate that mayors revise beliefs about early childhood programs when presented with credible evidence, although the pattern of updating differs from \textcite{vivalt_how_2023}. This literature establishes that beliefs are malleable and policy-relevant. Less is known, however, about whether and how a concrete top-down reform shifts implementers’ beliefs and their on-the-ground behaviors, and whether those two margins move in tandem or come apart.

This paper makes two contributions. First, it provides among the first causal estimates of how bureaucrats’ (teachers’) policy attitudes adjust in response to an actual reform, rather than to an experimental information treatment. Second, it examines the extent to which changes in stated attitudes translate into implementation behavior, thereby speaking to the relationship between expressed support and revealed actions in the policymaking process. My setting, teacher response to free primary education policy in a low-income creates a context where rents are plausibly at stake through placement into fee-free public schools, thus offering a credible test of whether bureaucrats’ stated endorsement can serve as a credible proxy for implementation fidelity.

Empirically, I assemble nationally representative survey data spanning periods before and after the FPE reform and combine them with measures of teachers’ policy attitudes and multiple indicators of behavior relevant to implementation. To identify reform effects, I employ a difference-in-differences design that exploits temporal variation around the policy coupled with cross-sectional comparisons to appropriate counterfactual groups. For attitudes, I compare changes for teachers relative to other groups not directly charged with school-level implementation. For behavior, I study household interaction with schools, specifically, whether eligible parents report paying a bribe to secure a child’s place. I also benchmark teacher's perception-based corruption measures against other occupational groups. 

My main results highlight a disconnect between teachers’ stated support for FPE and their behavioural response. Teachers’ support for FPE rises by 18 percentage points following the reform. Yet, this postive attitudinal shift coincides with increased rent-seeking behavior: over half of eligible parents report paying a bribe for school placement and public reports of teacher corruption increase by 32 percentage points. This pattern suggests a "stated preference-revealed preference" divergence such as those discussed by \textcite{beshears_how_2008}. In this setting, implementers endorse reform objectives while simultaneously engaging in behaviors that undermine equitable access.

The evidence points to a demand-side mechanism: the reform generated a surge in enrollment demand that, combined with capacity constraints, increased teachers' discretionary power over scarce school slots. This enhanced gatekeeping authority created opportunities for rent extraction despite teachers' stated support for the policy's egalitarian goals.

These findings have important implications for reform design and implementation. Stated bureaucratic support, while potentially necessary, is insufficient for ensuring implementation fidelity. Policymakers must anticipate how large-scale interventions reshape local incentives and alter the distribution of discretionary authority, particularly in weak monitoring environments. Reforms intended to expand access may inadvertently create barriers for intended beneficiaries if implementation challenges are not adequately addressed.

The remainder of the paper proceeds as follows. Section 2 describes the Tanzanian education context and FPE reform. Section 3 presents the data and empirical methodology. Section 4 reports results on attitude and behavior changes, including robustness checks and heterogeneity analysis. Section 5 discusses mechanisms and policy implications. Section 6 concludes.

\section{Tanzania's universal education reform}
TBD

\section{Methods and Data}


\subsection{Data}
I use nationally representative survey data from the Afrobarometer project for Tanzania. Specifically, I draw on Rounds 1--3 of the survey, which were implemented in 2001, 2003, and 2005, respectively. The Afrobarometer is a repeated cross-sectional survey designed to capture citizen attitudes, experiences, and evaluations of democracy, governance, and service delivery across African countries. In Tanzania, the sampling strategy follows a stratified, clustered, and nationally representative design, allowing inference at the population level.  

The period covered by these data is particularly well suited to study the effects of Tanzania's free primary education (FPE) reform on teacher beliefs and behaviors. While the policy was announced in 2001, it was officially implemented in 2002. This means that Round 1 provides a baseline measure of beliefs and behaviors just prior to implementation, while Rounds 2 and 3 capture responses in the immediate aftermath of the reform. 

\subsection{Empirical strategy}
To identify the effects of FPE reforms on both teachers’ beliefs and eventual behavioral responses, I estimate difference-in-differences (DiD) models. The first specification examines whether teachers update their support of the policy, while the second investigates whether teachers engage in rent-seeking behavior following the reform, thus undermining the policy reform itself.  

\subsubsection{Support for the policy}
To estimate whether teachers updated their support for the policy reform following its implementation, I estimate following model:  
\vspace{-1em}

\begin{equation}
    \label{eq:beliefs}
    \text{Support}_{id} = \alpha + \beta_1 \, \text{Teacher}_{i} 
    + \beta_2 \, \text{Post}_{t} 
    + \delta \, (\text{Teacher}_{i} \times \text{Post}_{t})
    + \mathbf{X}_{i}' \gamma 
    + \mu_{d} + \varepsilon_{id},
\end{equation}

where $\text{Support}_{id}$ is an indicator equal to one if individual $i$ in district $d$ supports free primary education. $\text{Teacher}_{i}$ is a dummy for being a teacher, $\text{Post}_{t}$ is an indicator equal to one for survey rounds conducted after the reform (2005), and $\mathbf{X}_{i}$ is a vector of individual-level controls (age and education). District fixed effects $\mu_{d}$ absorb time-invariant unobserved heterogeneity, and standard errors are clustered at the district level. The coefficient of interest is $\delta$, which captures the differential change in support for FPE among teachers relative to non-teachers after the reform.  

\subsubsection{Rent-seeking}
To examine teachers behavioral response to the FPE reform, I investigate whether they report an increase in rent-seeking behavior. I estimate an analogous model that captures changes in survey participants perception of teacher corruption, relative to other public servants corruption. Using other public servants' reported corruption (for example, police officers, judges etc) as a counterfactual helps to absorb general corruption trends within the country. I implement the following specification:  
\vspace{-1.5em}

\begin{equation}
    \label{eq:rentseeking}
    \text{Corruption}_{id} = \alpha + \theta_1 \, \text{Teacher}_{i} 
    + \theta_2 \, \text{Post}_{t} 
    + \kappa \, (\text{Teacher}_{i} \times \text{Post}_{t})
    + \mathbf{X}_{i}' \lambda 
    + \mu_{d} + \varepsilon_{id},
\end{equation}

where $\text{Corruption}_{id}$ equals one if individual $i$ in district $d$ reports that a public servant is corrupt. $\text{Teacher}_{i}$ is a binary representing whether the teacher is the public servant in question and 0 if other public servant. All other variables are defined as above. The coefficient of interest $\kappa$ measures the extent to which teachers are more likely to engage in rent-seeking in the post-reform period relative to the pre-reform period, relative to other public servants.  

Both specifications are estimated using weighted least squares with Afrobarometer sampling weights, and standard errors are clustered at the district level. 

\section{Background characteristics}

TBD


\section{Results}


\end{singlespace}

\newpage

\singlespacing

%This text contains \charactercount{main} characters.

\printbibliography[title={References}] \pagebreak 
\end{document}


